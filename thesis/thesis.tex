% !TEX encoding = UTF-8 Unicode
% !TEX spellcheck = en-US


% This is the root file of your thesis: thesis.tex
% A line starting with % is a comment. In some cases, I have included a command preceded by a %. You may activate the command by removing the %.

%%===================================
\documentclass[12pt]{report}
\usepackage{ramsstyle}
\usepackage{wrapfig}
\usepackage[dvipsnames,table,xcdraw]{xcolor}
\usepackage{todonotes}
\usepackage{enumitem}
% \usepackage[shortlabels]{enumitem}
\usepackage{listings}
\usepackage{float} % to set position of figures with [H]
\usepackage{pdfpages} % to include pdf
% tables
\usepackage{booktabs}
% \usepackage[table,xcdraw]{xcolor}

\usepackage{caption}
\usepackage{subcaption}
\captionsetup[figure,lstlisting]{font=footnotesize}

\definecolor{codegreen}{rgb}{0,0.6,0}
\definecolor{codegray}{rgb}{0.5,0.5,0.5}
\definecolor{codepurple}{rgb}{0.58,0,0.82}
\definecolor{backcolour}{rgb}{0.95,0.95,0.92}

\definecolor{bluekeywords}{rgb}{0,0,1}
\definecolor{greencomments}{rgb}{0,0.5,0}
\definecolor{redstrings}{rgb}{0.64,0.08,0.08}
\definecolor{xmlcomments}{rgb}{0.5,0.5,0.5}
\definecolor{types}{rgb}{0.17,0.57,0.68}
% \lstdefinestyle{mystyle}{
%     backgroundcolor=\color{backcolour},   
%     commentstyle=\color{codegreen},
%     keywordstyle=\color{magenta},
%     numberstyle=\tiny\color{codegray},
%     stringstyle=\color{codepurple},
%     basicstyle=\ttfamily\footnotesize,
%     breakatwhitespace=false,         
%     breaklines=true,                 
%     captionpos=b,                    
%     keepspaces=true,                 
%     numbers=left,                    
%     numbersep=5pt,                  
%     showspaces=false,                
%     showstringspaces=false,
%     showtabs=false,                  
%     tabsize=2
% }
% \lstset{style=mystyle}
\lstset{language=c++,
captionpos=b,
%numbers=left, %Nummerierung
%numberstyle=\tiny, % kleine Zeilennummern
frame=lines, % Oberhalb und unterhalb des Listings ist eine Linie
basicstyle=\ttfamily\footnotesize,
showspaces=false,
showtabs=false,
breaklines=true,
showstringspaces=false,
breakatwhitespace=false,
numberstyle=\tiny\color{codegray},
numbers = left,
tabsize = 2,
escapeinside={(*@}{@*)},
commentstyle=\color{greencomments},
morekeywords={partial, foreach, var, value, get, set, null, false, true},
keywordstyle=\color{bluekeywords},
stringstyle=\color{redstrings},
}


% \hypersetup{%
%   colorlinks=false,% hyperlinks will be black
%   urlbordercolor=gray,
%   linkbordercolor={0 0 0},
%   citebordercolor=gray,
%   filebordercolor=gray,
%   pdfborderstyle={/S/U/W 1}% border style will be underline of width 1pt
% }

\setlength{\marginparwidth}{3cm}
% \setlength{\parindent}{0pt}
%%===================================
%Write the various parts of your thesis as separate files and include them into the main file by the command \include{name of included file}. When you compile the LaTeX file, you may choose which subfiles to include by the command

%\includeonly{chapter01,chapter02}

%%===================================

\begin{document}
% % !TEX encoding = UTF-8 Unicode
%!TEX root = thesis.tex
% !TEX spellcheck = en-US

%This is the Titlepage
%%=========================================
\thispagestyle{empty}
\includegraphics[scale=1.1]{fig/NTNU}
\mbox{}\\[6pc]
\begin{center}
\Huge{Towards a 3D model of the rat brain in AR as an educational tool}\\[2pc]

\Large{Ole Ravna}\\[1pc]
\large{December 2020}\\[2pc]

PROJECT THESIS\\
Department of Computer Science\\
Norwegian University of Science and Technology
\end{center}
\vfill


\noindent Supervisor 1: Ekaterina Prasolova-Førland

\noindent Supervisor 2: Gabriel Kiss 

\noindent Supervisor 3: Menno Witter

 % This is the titlepage
\setcounter{page}{0}
\pagenumbering{roman}
% !TEX encoding = UTF-8 Unicode
%!TEX root = thesis.tex
% !TEX spellcheck = en-US
%%=========================================
\addcontentsline{toc}{section}{Preface}
\section*{Preface}


Something, something... \\[2cm]

\begin{center}
Trondheim, 2020-12-16\\[1pc]
(Your signature)\\[1pc]
Ole Ravna
\end{center}
% % !TEX encoding = UTF-8 Unicode
%!TEX root = thesis.tex
% !TEX spellcheck = en-US
%%=========================================
\addcontentsline{toc}{section}{Acknowledgment}
\section*{Acknowledgment}
I would like to thank the following persons for their great help during \ldots


\begin{flushright}
O.R.\\[1pc]
% (Your initials)
\end{flushright}





% !TEX encoding = UTF-8 Unicode
% !TEX root = thesis.tex
% !TEX spellcheck = en-US
%%=========================================
\addcontentsline{toc}{section}{Abstract}
\section*{Abstract}

% The following rapport is part of TDT4501 Specialization Project

This study aims to explore whether Augmented Reality can be used as a tool for medical students learning neuroanatomy. A application, \textit{Nevrolens}, was created with features approximating a conventional rat brain dissection, as well as a collaboration tools for students and educators to cooperate to simulate a lecturing environment. 
The thesis will explore the implementation of such a system, as well as how it performs in educational settings, and its possible use as a tool for remote learning.
While the results from this study are limited, they indicate the application to be of value in educating lower level medical students and that a AR system of this scope can be simple in use even for users with no prior experience with AR devices.
\newline

\textit{
\textbf{Keywords:} Augmented Reality, Mixed Reality, Education, Collaboration, Neuroanatomy, Remote learning
}

% the application has shown promise in its use as a virtual simulation of a rat brain dissection.
\tableofcontents
\setcounter{page}{0}
\pagenumbering{arabic}
% !TEX encoding = UTF-8 Unicode
% !TEX root = ../thesis.tex
% !TEX spellcheck = en-US
%%=========================================
\chapter{Introduction}

% {
%     \color{BrickRed}
%     I feel I need the two following sections before the Motivation sections, as they are quick introductions to the basics. But maybe there is a better way?
% }
%%=========================================


\section{Motivation}

% Computer based learning has been shown to increase understating 
% Neuroanatomy is a spatially complex field of study, requiring

Augmented reality is a technology which has experienced great leaps in recent years, and this growth has inspired many visons of medical potentials for this young technology. 
Within medical education there are many fields where visual understanding is critical, one of being neuroanatomy. Neuroanatomy is a highly complex domain both visually and spatially, the ability to use the human senses in a real-world setting could result in greater intuition and understanding. With that in mind the use of augmented reality could be a natural way to virtualize the experience of a brain dissection, and further the unique capabilities of AR could enable innovative ways of learning. \citep{Moro2017} shows the possibility of greater immersion and engagement while using augmented reality in teaching anatomy to medical students. This has also recently been shown with promising result by  \citep{Wish2020}, where COVID-lockdown required from-home teaching, and the use of HoloAnatomy, an anatomy application for the HoloLens, performed significantly better than even conventional in-class lectures.



% With this growth, the medical potential of the technology has seen an equal rise.
% inspired in medical professionals.
% With this [\dots], the envisioned medical potential of it use has 
% The technological growth has come spured 
% Growing hand in hand with this technology has 
% Jointly with this technological growth 

%%=========================================
% \section{Motivation}
% % This chapter should explain what the problem is, and why it should be fixed
% % Very quick intro
% % The problem
% % The possibility of a solution
% % Maybe rewrite the whole thing with this in mind?

% In light of the problems with physical brain dissections it is natural that the use of digital tools, three-dimensional modeling and visualization has been seeing growing use for educational purposes. 

% %%=========================================
% According to \citep{Dalgarno2010} computer-aided learning generally increases understanding for anatomy. As anatomy in general, and neuroanatomy specifically are highly complex domains both visually and spatially, the ability to use the human senses in a real-world setting could result in greater intuition and understanding. With that in mind the use of augmented reality could be a natural way to virtualize the experience of a brain dissection, and further the unique capabilities of AR could enable innovative ways of learning. \citep{Moro2017} shows the possibility of greater immersion and engagement while using augmented reality in teaching anatomy to medical students. This has also recently been shown with promising result by  \citep{Wish2020}, where COVID-lockdown required from-home teaching, and the use of HoloAnatomy, an anatomy application for the HoloLens, performed significantly better than even conventional in-class lectures.

%%==========================================

\section{Problem Formulation}
% Define the problem


%%=========================================
The main problem with most academic implementations, like \citep{Wish2020}, of AR in medical education is the use of head-mounted display (HMD) devices like the HoloLens 2 and Magic Leap, which in the near to mid-term future will have limited practical use in education, as a result of the high price-tag, combined with the still inadequate general use-case for these types of devices.
This project will try to mend these challenges by having the lecturer using an HMD and having student view and interact with the lecture in an AR-based application running on their smartphone. 
%%=========================================
This is possible because of the great leap in AR-performance seen in recent models of Android and especially iPhones, in combination with development platforms like \nameref{chap:unity}, \nameref{chap:mrtk} and \nameref{chap:photon} which enables multiplatform development and real-time collaboration between devices. 
% In this pursuit we will also make use of high-resolution 3D imagery of a rat brain (see \autoref{chap:ratbrain}).  
% %%=========================================
% \todo[inline]{Introduce Nevrolens and WHS brain}
The aim of the project will be to create a seamless educational experience in Augmented Reality which can be valuable both on an HMD device and a modern smartphone. The focus will be on investigating its feasibility as an educational tool both in a lecture-type setting and for students to explore the brain anatomy independently. 


%%=========================================
% \subsection*{What Remains to be Done?}

%%=========================================
\section{Research Questions}
What follow are the research questions which motivates this project: \\

\noindent
% \paragraph*{Main RQ:} How can AR support teaching of rat brain anatomy and dissection for medical students?

\begin{enumerate}[label=\textbf{Main-RQ:}, left=\parindent]
    \item \label{mainrq} How can AR support teaching of neuroanatomy and dissection for medical students?
\end{enumerate}

\begin{enumerate}[label=\textbf{Sub-RQ\arabic*:}, left=\parindent, labelindent=1em]

    \item \label{subrq1} How should interaction in be implemented in AR to accommodate medical students and educators?
    \item \label{subrq2} How will a collaborative experience shared between an HMD and a smartphone compare to accommodate medical users?
    \item \label{subrq3} Can understanding be increased by integrating microscopical data into a macroscopical model?

\end{enumerate}


% \begin{itemize}
%     \item {
%         \textbf{Sub-RQ1:} How should interaction in be implemented in AR to accommodate medical professionals?
%     }
%     \item {
%         \textbf{Sub-RQ2:} How will a collaborative experience shared between an HMD and a smartphone compare to accommodate medical professionals?
%     }
%     {
%         \newline
%         \color{BrickRed}
%         \textbf{Sub-RQ3: }
%         Something about macro + microscopic visualization, some suggestions:
%         \item Can microscopical data seamlessly be integrated into a macroscopical model? 

%         \item Can understanding be increased by integrating microscopical data into a macroscopical model?

%         \item Will having integrated microscopical data in a macroscopical model lead to greater understanding?
%     }
% \end{itemize}


%%=========================================
\section{Approach}


\subsection*{Research method}

\begin{wrapfigure}{r}{0.45\textwidth}
    \begin{center}
        \includegraphics[width=0.42\textwidth]{fig/researchplan_image}
    \end{center}
    \caption{Model of the research process as illustrated in \citet{oates2006}}
    \label{researchplan_img}
\end{wrapfigure}

The research questions were derived through discussing the needs of the intended users with neuroscientists at the Kavli Institute. It was then narrowed down by a literature review, finding a lack of satisfactory substitutions for real brain dissections and especially finding no attempt at a practical multiplatform application for a more scalable use for students. The projects research question falls under the strategy of Design and Creation as the main goal is to develop a useful application for medical education. The focus on a smartphone solution was further motivated by the COVID-pandemic making from-home learning quite essential and making the passing around of HMD devices an unwanted scenario. As part of an agile software development model the gathering of qualitative data from observations and interviews within the scope of user testing will be essential. 

% \subsection*{Development method}

\section{Contributions}
%% write about macro vs micro stuff

The research product resulting from this project will be a new computer-based software application using augmented reality and running on multiple platforms like HoloLens 1 and 2, Android and more. The aim will be to develop an application that can bridge the gap between expensive head mounted displays and everyday smartphones which you will find in the pocket of any student, and to use this as a collaborative tool for learning neuroanatomy. Throughout the development period we will consult with medical professionals and gather feedback from students on the usability of the application.

%%=========================================
% \section{Outline}



\include{chapters/Background}
\chapter{Requirements}\label{chap:req}


% \section{Requirements}

% cadavers difficult to get, vr -> less awareness
%
%


The first meeting initializing the project took place at VRLab Dragvoll in early September, here I was introduced to the general background and the problem description of how the neurologist stakeholders envisioned the use of AR for neuroanatomical education. It was explained how cadavers for education are difficult to acquire and therefore used quite sparingly. 
Another problem we discussed was related to the difference in medium between VR and AR. While the application \nameref{chap:vrvis} did have many of the features envisioned, and could have been a basis for further development. The fact that is was implemented in VR was problematic for the envisioned use cases. Being completely enclosed visually limits its use case in lectures and in any use case with collaboration in a physical space. Generally the loss of spatial awareness and eye contact as a result of using VR headsets was though of as an impediment for using VR for such an application. 
% something about the data set?
Thus, we had an outline of a neuroanatomical education tool in AR using the HoloLens 2 and concluded with some questions and requirements for the project:

\begin{enumerate}\label{mennoslist}
    \item Can the current VR dataset\footnotemark be used in the HoloLens 2 AR environment?
    \item If not, which steps need to be taken to use the segmented WHS rat brain to develop a suitable 3D model that can be used in AR?
    \item Develop an optimal user interface for a single person to explore the rat brain as if the user is doing a dissection of a real brain.
    \item Develop/test ways to make this a multiuser/shareable tool adequate in a teaching environment.
    \item Explore ways to integrate microscopical data into the AR representation.
    \item Describe/explore the feasibility to implement the system for Human neuroanatomy education.
\end{enumerate}

\footnotetext{Referring to \nameref{chap:vrvis} by \citep{Elden2017}.}

Here items 1-4 were deemed critical for the project, while 5 and 6 were dependent on the progress made.

This meeting together with the list formed a clear problem description and can be seen as the initial discovery process of the project. Though the following period of exploring the newly arrived HoloLens 2 and its capabilities, we formed a set of \textit{system requirements}. 
System requirements are descriptions of how a system should operate, what it should be able to do and the constraints of its operation. The requirements must reflect the stakeholders needs for the system \citep{PUboka}. System requirements are generally split into functional requirements, which describe specifics of what the system (and its sub-systems) should do, and non-functional requirements, which generally are descriptions of the user experience of the system as a whole. 
What follows are the system requirements decided on for the application: 

\subsection*{Functional Requirements}
\begin{enumerate}
    \item {
        \textbf{Implement a brain dissection tool in AR.}\\
        The app should render a brain at sufficient quality for educational use, and have the tools for creating a dissection experience in AR.
        
    }
    \item {
        \textbf{The application must run in HoloLens 2 and at least one mobile platform}\\
        The ability to run a version for the app on multiple platforms is essential for the purpose of this project. While the main platforms are HoloLens and mobile, others may also be implemented in the future. 
    }

    \item {
        \textbf{Implement cross-platform collaboration over network}\\
        For the application to have value above a single user it is important that it can be used with a HoloLens and a more accessible platforms in a collaborative manner. 
    }


\end{enumerate}

\subsection*{Non-Functional Requirements}

\begin{enumerate}
    \item {
        \textbf{Medical students should find educational use for the app.}\\
        It is critical that there is educational value in the application. 
    }
    \item {
        \textbf{The application should be usable without outside guidance.}\\
        The app should have a clear and understandable design, such that a new user should be able to navigate the app by them self, even with minimal experience with AR.
    }
    \item {
        \textbf{All relevant usability criteria for a mixed reality app should be met.}\\
We should work to not fall under the 'meets' criteria on any relevant metric in the App quality criteria\footnote{https://docs.microsoft.com/en-us/windows/mixed-reality/develop/platform-capabilities-and-apis/app-quality-criteria}. This includes criteria on; FPS, spatial anchoring and view comfort. 
    }
\end{enumerate}
\chapter{Technical Design}

This chapter will give a overview of the structure of the application as well as some choices taken when developing the research product, Nevrolens. 

\subsection*{Unity Scene Graph}

Within Unity a \textit{Scene} consists of a \textit{scene graph} which is a tree structure of \texttt{GameObjects}. By default a scene consist of a \texttt{Directional Light} lighting up the scene at its default light source and a \texttt{Main Camera} which is the view point of the running game. In addition, the MRTK library will add two objects to the scene graph, one called \texttt{MixedRealityToolkit} which contains configuration of the Mixed Reality features and systems. This is where input systems are defined and where control of spatial awareness and boundary detection is handled, in short all features and sensors of the HoloLens system or other AR system are defined and controlled here. The other object added by MRTK is the \texttt{MixedRealityPlayspace}, this encapsulates the \texttt{Main Camera}, but is lacking any useful documentation on what its purpose is. The name could be hinting at it being the parent of the \textit{Playspace}, meaning all \texttt{GameObjects} in the game. However, even MRTK demos seem to ignore this object and thus it has not been used in this project either.

The functionality of the scene graph, other than organizing GameObjects, is that child objects inherit the position, rotation and scale of their parents, thus simplifying transformation of complex object. This naturally structures many systems, however in a AR application there can be many independent 3D objects floating in space. In addition, some objects are dependent not on their parent, but on a defined object \texttt{Transform}. Therefore, some organization is needed and some objects are placed by choice and convenience rather than any practical reason. Another practical use for child objects are the use of the \texttt{GameObject.GetComponentsInChildren()} and \texttt{GameObject.GetComponentInParent()} methods which allows for simple access to \texttt{Components} in child and parent \texttt{GameObjects}, this is however of limited use as such dependencies in code has a tendency to result in tedious bugs.

The top most application specific object of the project is the \texttt{BrainSystem}, this acts as the parent GameObject for all objects defined by the application. The right side of \autoref{fig:brainsystem} gives an overview of all 3D object in the \texttt{BrainSystem}. The \texttt{InfoBoard} on the right, the button group, or \texttt{HandMenu} in the center and the complete brain model with axes etc. named \texttt{GameWorldSpace}, are spatially independent systems all having \texttt{BrainSystem} as a parent, this can also be seen in \autoref{fig:brainsystem} in the scene hierarchy on the left side. The reasoning for having the parent object \texttt{BrainSystem} is purely to to tidy up the top layer of the scene graph and having a clear distinction of project specific custom objects. 

The main attraction within the \texttt{BrainSystem} is the \texttt{GameWorldSpace}, it is the parent of the brain model and all objects with are spatially dependent on the brain. This allows for movement and scaling of the whole model worldspace. This is also the local space of the synchronized multiplayer world. 

% notice that the blue "black board" on the right, the button group and the brain are independent spatially.

\begin{figure}[ht]
    \centering
    \includegraphics[width=\textwidth]{fig/brainsystemoverview5.png}
    \caption{Every 3D object in the \texttt{BrainSystem}}
    \label{fig:brainsystem}
\end{figure}
% In general there is few limitations on how to structure the scene graph, its functionality other than structural,
\subsection*{Networking Solution}

Multiplayer games in Unity can be created in numerous ways, in the development phase of this project three solutions were explored; UNET,  LiteNetLib and Photon PUN2 . Common for all are that they are mature, reliable and are well documented, they all support multiple device types including all devices within the scope of this project, there are however some very clear differences making the choice for this project quite simple.
% mlapi, mirror, playfab
UNET is Unity's own default networking solution, it provides high level functionality and is generally easy to use. It is however deprecated and will be discontinued by the end of 2021, an open source fork of the networking API, called \textit{Mirror} has seen continuing development and improvement, but because of the state of the original project, both were deemed nonideal. Unity is working on a new networking solution called \textit{MLAPI}, it is in alpha stages but shows great promise.

LiteNetLib is a open source, and more low level framework. It is intended for use cases where in-depth control of the networking processer are wanted or needed, if high performance and low latency is important this would be a good choice. It supports peer-to-peer and self-managed servers. Because this project can be thought of a small scale proof of concept, it is of limited concern whether the networking is highly performant and seeing as a low level API is more complicated to implement it is neither a optimal use of a single developers time.

Lastly, Photon PUN2 is the an high level networking library with managed hosting and a free basic plan for up till 20 concurrent users. This makes it ideal for small projects and single developers. It is also the general first choice for networking solution in Unity and its surge in popularity was a reason for Unity abondonign 


% mature, reliable and easy to use with 


\chapter{Development Process}


\section{Software Process}
% https://www.wikipendium.no/TDT4140_Programvareutvikling/nb/#kapittel-2-software-processes

% Incremental development etc. \#agile
% \large{Where everything I learned in PU should shine!}

Even though I have developed the Nevrolens application by myself, I have strived to use best practices for a software development workflow. These practices have generally grown out of the the needs of a multi-developer setting, enabling simpler use of collaboration and version control. Though their value possibly increases exponentially by the number of team members, I have found value in the structure and clarity I find in the workflow. 

\begin{figure}[h]
    \includegraphics[width=0.33\textwidth]{fig/gitkraken_gitlog}
    \includegraphics[width=0.33\textwidth]{fig/mergerequests}
    \includegraphics[width=0.33\textwidth]{fig/release_gitlab}
    \caption{Feature branches, merge requests and releases.}
\end{figure}

My workflow is based on \textit{Gitflow}, a workflow framework optimized for continuous software development. In short, this is just a very basic rule set for branch-naming and the sanctity of the master-branch (requiring merge requests of only product ready code), within the version control system \nameref{chap:git}. It does however act as a fundament which enables practices like rapid release cycles, because of the clearly define production ready state, and the integration with lean development technics like Kanban. This stems from the parallels between feature-branches in Gitflow and the \textit{ticket} in Kanban. In practice, this means that tickets, with issues or new features for the app, are created on in the \textit{Backlog} column of the Kanban board and are then moved to \textit{Doing} column simultaneously as a feature-branch is created with the ID of the ticket, e.g. \texttt{feat/NL-42}. All of this is automated in the Git management tool \nameref{chap:gitkraken}, which manages both the git-repo and the Kanban board.

\begin{figure}[h]
    \includegraphics[width=\textwidth]{fig/kanban2}
    \caption{A snapshot of the Kanban board in GitKraken, after a development sprint, when completed tickets are archived (closed).\\ Note: \textit{Priority} acts as pined tickets on \textit{Backlog}, as the backlog tends to sizeable.}
    \label{fig:kanban}
\end{figure}

This workflow, by design, supports an agile development process. Agile approaches to software development are generally human-centered, valuing individuals and interactions over processes and tools\footnote{The Agile Manifesto https://agilemanifesto.org/}, and focused on iterating rather than upfront planning. This is ideas which are beneficial for single-developer or small teams especially when developing for new platforms like the HoloLens 2. 
While the project aims for an agile approach, the sprint cycle core to the agile development, where stakeholders are involved for regular feedback, has, due to a number of factors like COVID-19, only really been done for one cycle. However, the steps taken for an agile workflow should enable more agile development for the master project.

% One thing that I have not incorporated in to the workflow is user stories. The tickets in \autoref{fig:kanban} are written only for my understanding and will seem confusing and untidy for others. By using the concept of user stories 

% and will limited testing possibilities due to COVID-19 


% this enables me as a developer to test the product rapidly against users. 
% , this means having rules for branch naming, creating merge requests when merging to the `master`-branch, and 
% !TEX encoding = UTF-8 Unicode
% !TEX root = ../thesis.tex
% !TEX spellcheck = en-US
%%=========================================



\chapter{Implementation}

% Hololens, UWP, WMR


% and non-functional requirements, which describe  

\section{Development phase}

\subsection*{0th iteration: Initializing application, importing and simplifying brain model}

\begin{wrapfigure}{r}{0.45\textwidth} 
    \centering
    \includegraphics[width=0.45\textwidth]{fig/hololens2polycount}
    \caption{Figure showing frame rate as a function of polygon count on HoloLens 2. \\ Credit: \href{https://community.fologram.com/t/hololens-2-polygon-count-and-frame-rate/49}{Fologram}}
    \vspace{20pt}
    \label{fig:polycount}
\end{wrapfigure}

The first phase of development started by acquiring the surface model of the WHS rat brain created by \citet{Elden2017}. This was done by simply moving the FBX files from the \nameref{chap:vrvis} application and to a new Unity project. After initializing MRTK by following their \href{https://microsoft.github.io/MixedRealityToolkit-Unity/version/releases/2.2.0/Documentation/GettingStartedWithTheMRTK.html}{Getting Started documentation}, the application was ready to deploy on the HoloLens 2. This resulted in a barely running application as the polygon count of the brain model was orders of magnitude larger than what is recommended to maintain adequate performance on the HoloLens 2, which is in the order of 100,000 polygons shown in \autoref{fig:polycount}. The model used by Elden was scaled to run on workstation computer outputting to an HTC Vive, and thus his model was reduced from a original 16 million polygons to around 3 million. The HoloLens 2 runs all calculations on-device on a mobile ARM-based processing unit and naturally the brain model created for rendering on a dedicated workstation graphics card had to be further scaled down. 
This downscaling was first experimented with doing at run-time dynamically on-device using the library \textit{UnityMeshSimplifier}. It was quickly determined that this was not a viable solution both because of untenable processing time, but also because the simplified result had a huge impact on quality of model, hinting at the performance optimization that had to be done on the simplifier algorithm to be able to execute at run-time. The next and final solution for downscaling was to use the \textit{decimate} modifier in \nameref{chap:blender}. \textit{Incremental decimation} is a mesh simplification algorithm which trades some speed for higher mesh quality, in contrast to the \textit{vertex clustering} presumably used in UnityMeshSimplifier which prioritizes speed in such a way that topology is not preserved. Within Blender functionality for simple application of the modifier to all objects in a tree-structure was not found, or understood to exist, so a script applying the decimate modifier with a given ratio was written, see \autoref{item:blenderscript}. The \texttt{ratio} parameter is a value between 0 and 1, representing the scaling of the resulting mesh' polygon count.

\begin{lstlisting}[language=python, label={item:blenderscript}, caption={Blender script applying a decimate modifier to all relevant objects in a scene.}]
import bpy # importing the blender python library

def decimate(ratio, replace = True):
    # Finds all objects and filters irrelevant objects from the FBX 
    brainparts = [n for n in bpy.data.objects \
        if n.name not in ("Camera", "Light")] 

    for part in brainparts:
        mod = part.modifiers.new(type='DECIMATE', name='Decimate')
        mod.decimate_type = 'COLLAPSE'
        # Sets the specifies strength to the decimate operation. 
        mod.ratio = ratio
# Calls function with given decimate strength.
decimate(0.08)
\end{lstlisting}

The resulting decimated model, even at the ratio of 0.08, was visibly nearly indistinguishable from the original model when looking at them through the HoloLens 2 display which, as described in \autoref{chap:hololens2}, is somewhat blurry. \autoref{fig:decimate} shows the difference as seen in the Unity editor. Ultimately, a decimation ratio of 0.08 was chosen as a compromise between detail and performance being about 300,000 polygons, this compromise will be discussed further in \autoref{chap:discussperformance}. At this stage requirement 1 and 2 in the initial requirements from \autoref{chap:req} could conclusively be answer as possible and completed.
\begin{figure}[ht]
    \includegraphics[width=\textwidth]{fig/brainmodeldecimateratio2.png}
    \caption{WHS rat brain models with decreasing polygon count.}
    \label{fig:decimate}
\end{figure}



\subsection*{1st iteration: MVP}
Having a surface model of the brain running reasonably well on the HoloLens 2, the next step in developing the application was to implement basic AR-based interact features. The brain model consist of an empty parent object with 29 children each containing the mesh of a delineated brain structure, see \autoref{fig:brainunitytree}. Adding the \texttt{Object Manipulator} component from MRTK and a standard Unity \texttt{Mesh Collider} component to each child in the brain model allows for picking apart the brain. This is done by grabbing and moving each separate structure with a MRTK defined \textit{pointer}, this is the logical abstraction for the simplest interact handling with HoloLens 2 giving the user a virtual laser pointer from their finger. The resulting action can be seen in \autoref{fig:grabbrain}. 

% \begin{wrapfigure}{r}{0.38\textwidth} 
%     \centering
%     \includegraphics[width=0.35\textwidth]{fig/brainunitytree.png}
%     \caption{The tree structure of the Unity \texttt{GameObject} of the brain model.}
%     \vspace{-10pt}
%     \label{fig:brainunitytree}
% \end{wrapfigure}

\begin{figure}[ht]
    \centering
    \includegraphics[width=0.30\textwidth]{fig/brainunitytree.png}
    \includegraphics[width=0.60\textwidth]{fig/shittyassbraintreediagram.png}
    \caption{The tree structure of the Unity \texttt{GameObject} of the brain model.}
    \label{fig:brainunitytree}
\end{figure}

\begin{figure}[ht]
    \centering
    \includegraphics[width=0.8\textwidth]{fig/grabbrainsection.png}
    \caption{Grabbing the neocortex brain structure with a MRTK pointer in the Unity editor.}
    \label{fig:grabbrain}
\end{figure}

An apparent problem at this stage was that thought the brain structures are separate objects, they were difficult to visually distinguish from each other. A script which took all child objects and applied a random color to each was written and placed on the parent object, thus quickly giving some visual separation of the structures. While implementing this feature, the \textit{material} of each child was changed from Unity's default material to a \textit{MRTK Standard} material. Materials are the way Unity handles rendering details for each object, this is where shader, texture and general rendering options are configured. The MRTK Standard materials is a set of materials using the the \texttt{MixedRealityStandard.shader} shader, this shader is optimized for MR use, and superficially for HoloLens, and is meant for fulfill all shader-needs when developing for these platforms. 

With a some basic visibility and manipulation features for the brain model, the next natural step was tackling the system requirements, specifically the first functional requirement, implementing brain dissection. A \textit{clipping} shader was written and implement to work with the brain, giving more control over the feature than using MRTKs prebuild clipping feature, but seeing as it was not possible to combine a custom shader with MRTK optimizes feature set for AR rendering, the custom clipping implementation was abandoned in favor of MRTK, using the aforementioned MixedRealityStandard shader. Clipping has the effect of removing vertices by some defined function, and by using a prebuilt clipping plane prefab and declare on which meshes is should act, a dissection affect was created. A handle for manipulating the plane was added for ease of use, by dragged a ball the plane would move such that it was a fixed distance from the ball and perpendicular to the line between the ball and the center of the brain. 


Further, a hovering menu displaying the name of the last grabbed brain structure and buttons for the actions moving, transparency and dissection was implemented. This was created by modifying a MRTK prefab and updating its name based on the name of the \texttt{GameObject} the \texttt{pointer} targeted while dragging, at the same time a selection lighting effect as applied by simply enabling \texttt{Border Lighting} in the MixedRealityStandard shader. Unity's layer functionality was used to ensure that it was a brain structurer being dragged. One last feature implemented at this phase was a tap-to-spawn feature, this entailed using the \texttt{pointer} to tap on the physical space, and using spatial awareness to place the brain at the locations the the user tapped. In MRTK spatial awareness is enabled by default and its mesh can be identified by a predefined Unity layer, thus \autoref{item:sudopointer} shows a simplified implementation of the \texttt{EventHandler} method, \texttt{OnPointerDown} which spawns the brain if the pointer is hitting the spatial awareness mesh and enables border lighting and menu text if it hits a brain structure.

\begin{lstlisting}[language=c, label={item:sudopointer}, caption={A simplified version of the event function called when a \texttt{Ponter} is clicked.}]
    void OnPointerDown(MixedRealityPointerEventData eventData)
    {
        if (!HasTarget(eventData.Pointer)) 
            return;
        Vector3 hitPoint = GetHitPoint(eventData.Pointer);
        GameObject target = GetCurrentTarget(eventData.Pointer);

        switch (target.layer)
        {
            case SpatialAwarenessLayer:
            {
                if (BrainHasNotBeenSpawned())
                    SpawnBrainAt(hitPoint);
            }
            case BrainStructureLayer:
            {
                if (selectedStructure != null)
                    DisableBorderLighting(selectedStructure);
                EnableBorderLighting(target);
                SetMenuText(target.name);
                selectedStructure = target;
            }
        }
    }
\end{lstlisting}

The application was deployed for HoloLens 2, and was a first MVP demo of the research project. \autoref{fig:mvpdemo} shows spawning of the brain model from image 1 to image 2 in the top row, notice the pointer on the table in image 1. Image 3 illustrates the clipping feature, while image 4 has a user taking out the \textit{cornu ammonis 3} brain structure.

\begin{figure}[h]
    \includegraphics[width=0.5\textwidth]{fig/mvpdemo1.png}
    \includegraphics[width=0.5\textwidth]{fig/mvpdemo2.png}
    \includegraphics[width=0.5\textwidth]{fig/mvpdemo3.png}
    \includegraphics[width=0.5\textwidth]{fig/mvpdemo4.png}
    \caption{The first demo of the application.}
    \label{fig:mvpdemo}
\end{figure}

\subsection*{Next iterations: Continuing development}

The continuation of the project will be explored further, but will focus on implementation of highlighted features and and high-level overview of the process, rather than a chronological log as in previous sections. 
% This section will give a overview of the continued progress in developing the application and highlighting some specific features. 

The first demonstration of the application was done over video conference, with a pre-recorded YouTube video, demonstrating the features of the application.

\subsubsection*{UI, menus and stuff }

\subsubsection*{Feedback from demo}
% list view

\subsubsection*{Coloring brain}

\subsubsection*{Snapping}

\subsubsection*{Porting to Android}
The application was originally developed for HoloLens 2, but other platforms were always in mind and because of the use of \nameref{chap:mrtk}, deploying to other platform was relatively easy, within the documentation for MRTK, there was guides for how to build for Android. Interaction on Android is more complicated though. Because all interaction happens on screen additional effort has to be laid down to implement a good user experience on smartphone.  


\subsubsection*{Volumetric dissection plane}



\subsection*{3rd iteration: Implementing network}
\subsection*{4th iteration: Final product}










\chapter{Deployment}

The application in this research has been development using Microsoft's Mixed Reality Toolkit and has followed the platforms best practices and recommendations. The deployment process for HoloLens devices are based on sideloading of the application compiled and built with Visual Studio Community 2019, this is the standard way of both building and deploying for HoloLens. For Android the principle of sideloading is also used, but here Unity is responsible for compilation and the build process. 
The application as 


\subsection*{Installation of Nevrolens}

This section will be a guide for installing Nevrolens on HoloLens 2 and Android from packages hosted on GitLab.
By following the instructions for HoloLens 2, this should also work for HoloLens 1, but has limited testing on that platform as it was not a focus of this research.

\subsubsection*{Deploy to HoloLens 2}

\begin{enumerate}
    \item {
        \textbf{Go to the release page}\\
        Found here: \url{https://gitlab.stud.idi.ntnu.no/olevra/nevrolens/-/releases}
    }
    \item {
        \textbf{Choose a release}\\
        Preferably the topmost and latest. This research ends on is version 0.3.3. 
    }

    \item {
        \textbf{Download the HoloLens zip package}\\
        Under \textit{Packages} click on the package for HoloLens (1 and 2) to download it.
    }

    \item {
        \textbf{Unzip file}\\
        Open the downloaded ZIP-file and extract it.
    }

    \item {
        \textbf{Open the Windows Device Portal for HoloLens}\\
        Guide by Microsoft: \href{https://docs.microsoft.com/en-us/windows/mixed-reality/develop/platform-capabilities-and-apis/using-the-windows-device-portal}{Using the Windows Device Portal}
    }
    \item {
        \textbf{Install appxbundle}\\
        Under \textit{Views / Apps} click \texttt{Choose File} and locate the APPXBUNDLE-file inside the folder extracted from the ZIP-file. Then click \texttt{Install}.
    }

\end{enumerate}

\subsubsection*{Deploy to Android}

\begin{enumerate}
    \item {
        \textbf{Go to the release page on your Android device}\\
        Found here: \url{https://gitlab.stud.idi.ntnu.no/olevra/nevrolens/-/releases}
    }
    \item {
        \textbf{Choose a release}\\
        Preferably the topmost and latest. This research ends on is version 0.3.3. 
    }

    \item {
        \textbf{Download the Android APK-file}\\
        Under \textit{Packages} click on the package for Android to download it.
    }

    \item {
        \textbf{Open the downloaded file.}\\
        Your device will ask for your permission to install an application from a unknown source. Accepting this, the device will start installing the application.
    }

\end{enumerate}

\subsection*{Getting started with developing the project}
This section will briefly explain how to set up the project for development, this differs from deployment in that the goal is to be able to continue development of the project from within Unity and with supporting tools. This is the process the current developer uses when developing from a new computer.
Having completed this set up deployment of the application can be done, directly from Unity for Android and through Visual Studio 2019 for HoloLens devices.


\textbf{Requirements}
\begin{enumerate}
    \item Git and Git LFS
    \item { Unity 2019.4 LTS \\
    When installing add: \textit{Android Build support} }
    \item {Visual Studio 2019 \\
    When installing add: \textit{Universal Windows Platform development}, \textit{USB Device Connectivity}, \textit{Game development with Unity} }
\end{enumerate}
Begin by cloning the \textit{Git} repository by typing: \\ \texttt{git clone https://gitlab.stud.idi.ntnu.no/olevra/nevrolens.git} in to \textit{Git Bash}, than type \texttt{git lfs install \&\& git lfs fetch \-\-all}. Now open Unity Hub and locate and add the repository just cloned as a Unity project. Open it with Unity 2019.4, it is critical that this version is used for MRTK to work properly. First time opening this project will take some time, when it is done the project is ready.
% The deployment process can be found in MRTK documentation 
\chapter{Testing}


This chapter will explore how testing has been accomplished in this research project, both day-to-day software testing, but also feedback from stakeholders and formal user testing. Because of the current COVID-19 pandemic, physical meeting and user testing been done sparingly, and have at times been impossible carry out, in fact the fall semester did not see any physical testing by users not affiliated with the VRLab at NTNU Dragvoll. What follows are the approach and structure of testing, research results and feedback from the testing sessions will be expanded further in the \nameref{chap:results} chapter.

\section{Software testing}
During development, software testing has been done unstructured, mostly by way of regular deployment and on-device feature testing. This is admittedly not the most comprehensive testing system and can not guarantee intended behavior as systems are combined and restructured, in the same way as a test suit with \textit{unit testing} and \textit{integration testing} could. But it allows for more rapid development, less overhead for a single developer and can not be said to have meaningfully limited the research project. A case could be made for a testing suit being helpful in the continuation of the research as new developers will have a concrete indication of intended behavior, this has and could not have a high priority as a development goal, because of limitations in development time and the demand for creating research results.

\section{User Testing Precautions}

\subsection*{Consent}
Every user tester where handed a consent form at the beginning of each session. This assured the privacy rights of the test persons, and gave approval to use the findings of the test session in this research. The consent form can be found in \autoref{appendix:consent}.

\subsection*{Hygienic Measures}
Due to the COVID-19 pandemic, many new precautions have been enacted to prevent the spread of virtual pathogens. The most effective measure has undoubtedly been the limited number of testing sessions, even so what follow are the precautions taken during those physical meetings. 
Firstly, the general national guidelines of social distancing and clean hands were kept. 
Between each new person using a head mounted HoloLens 2, the headset was placed in a \textit{Cleanbox}, this is system specially design to disinfect AR and VR devices utilizing UVC radiation to destroy viruses. It claims to kill Covid-19 viruses with 99.999+\% efficacy\footnote{https://cleanboxtech.com}. Further, contact points and buttons on the device are swiped with alcohol-based sanitizer. The reason this step was not enough and the UVC box was needed is because of the fragile nature of the lenses of the AR device, it is generally recommended not to touch the lenses, and alcohol would naturally not be good. On Android devices this was however the main disinfection method as cellphones could be completely wiped. Lastly, the user would ware a hygienic mask under the headset which reduced contact between the user and the headset.
All in all these actions in combination with strict adherence to the national guidelines of physical distancing, washing and sanitizing of hands and waring of face masks were done to prevent spread of the virus. 

% zorro masks, UVC radiation, anti bacterial desinfactant, munnbind og avstand

\section{Stakeholder meetings}
Throughout the project, there have been multiple meetings with stakeholders from the Kavli Institute at St.Olavs. These have been a combination of physical meetings and virtual video conferences using Zoom. When meeting virtually the research prepared one or more video demonstrations captured on-device, either HoloLens 2 or Android, which were uploaded to YouTube for convenience. When meeting physically either the researcher or the stakeholder would use the application with live view enabled such that the other could see and comment on the usage. Afterwards, the progress was discussed and evaluated and feedback was given on usability and features. Primarily, the resulting feedback was in the form of feature request, and general technical in nature.
% Because of differing backgrounds the researcher and stakeholders had some communication issues,

\section{User Testing}
During the final stages of the research project, two user testing sessions were held. This both gave useful insight into what the application did well and badly. Many of the issues discovered in the first test session were improved upon, this process is explored in \autoref{chap:finaliter}. The last test session was at the end of the development phase on this project and was meant mainly to gather data for research. No further development has happened after this point and as such the last test session gives an accurate picture of the resulting software product of this research project, both in its proficiencies and its limitations. 

\begin{figure}[ht]
    \includegraphics[width={\textwidth}]{fig/usertesthololivestream.jpg}
    \caption{Tester using HoloLens 2, being lectured by the neurologist whos watching their actions through live feed from Device Portal.}
    \label{fig:usertesthololivestream}
\end{figure}

\subsection*{First testing session}

The first user testing had three participants, all being medical student. Two first year and one third year student. Additionally, one neurologist from the Kavli Institute was present in the role both of stakeholder in the research project and also as lecturer to the student whom all took courses held by the neurologist. This session did not test the Android application at all and only focused use with the HoloLens 2, this created a challenge for collaboration testing as only two HoloLens 2 devices were available. 

The session began by having the participants take a test to survey their knowledge of the rat brain. This test was a standard test from course work in medical courses held by the stakeholder neurologist, and is meant to measure the short term learning from a single lecture. Therefore the exact same test was taken after the end of the session when the participants had used the research application.

After taking this test the participants were freely testing the application somewhat unstructured, and familiarizing them self with the usage of AR devices which none of the participants were accustomed with. Throughout the usage a pattern naturally evolved such that the neurologist lectured students with headset on. This started when the neurologist and a single student both used the application through the headset, the neurologist guided the student and showed them the different brain structures and explained there purpose. The one on one nature of these lectures were broken up by having the lecturer view the live stream of two students through a Windows desktop computer, this happened mostly because of time constraints as it lent itself to both of the remaining students having the lecture rather than just one. 
Both situations were insightful as observations of practical use of the application, and were not designed, nor intended by the researcher. 


% As this one on one lecturer continued it became apparent 
% While the neurologist used the headset in the beginning, which resulted in a one to one lecture s
% and was somewhat unstructured in its execution. This resulted in 

\subsection*{Second testing session}



\begin{figure}[ht]
    \includegraphics[width={\textwidth}]{fig/usertestmennobeingmenno.jpg}
    \caption{Neurologist lecturing Android test users.}
    \label{fig:usertestmennobeingmenno}
\end{figure}


The second test session marked the end of this research projects development stage, thus it was a test of the final product in this research. Therefore, in addition to this test session being used for data gathering, it can also be seen as a final evaluation of the application. 

The participants in this session were five medical students and two computer science student 


% !TEX encoding = UTF-8 Unicode
%!TEX root = thesis.tex
% !TEX spellcheck = en-US
%%=========================================



\chapter{Results}\label{chap:results}

% % The result of this project is the preliminary work for my master thesis next semester, thus 

% \section{Nevrolens}\label{nevrolens}
% Nevrolens is the name I've given the application which is the research product of this project. It's an artistic (read; incompetent) combination of the Norwegian word Nevroanatomi and HoloLens. It is a AR application running on HoloLens 1, HoloLens 2 and Android. 

% The application is focused on delivering a single user experience, with features as cutting planes, scaling, moving brain parts and transparent brain parts. \autoref{fig:nevrolens_holo} show these features running on HoloLens 2, while \autoref{fig:nevrolens_android} show them on Android.
% It is packages and released on GitLab at \url{https://gitlab.stud.idi.ntnu.no/imtel/nevrolens}. 

% \begin{figure}[h]\label{fig:nevrolens_holo}
%     \includegraphics[width=0.5\textwidth]{fig/nevrolens/twohandedzoom.jpg}
%     \includegraphics[width=0.5\textwidth]{fig/nevrolens/clipping.jpg}
%     \includegraphics[width=0.5\textwidth]{fig/nevrolens/palmmenu.jpg}
%     \includegraphics[width=0.5\textwidth]{fig/nevrolens/brainpartsout.jpg}
%     \caption{Nevrolens v0.1.3 on HoloLens 2}
% \end{figure}


% \begin{figure}[h]\label{fig:nevrolens_android}
%     \includegraphics[width=0.5\textwidth]{fig/nevrolens/android_zoom_large.jpg}
%     \includegraphics[width=0.5\textwidth]{fig/nevrolens/android_clipping.jpg}
%     \includegraphics[width=0.5\textwidth]{fig/nevrolens/android_palmmenu.jpg}
%     \includegraphics[width=0.5\textwidth]{fig/nevrolens/android_partsout.jpg}
%     \caption{Nevrolens v0.1.3 on Android}
% \end{figure}





% \section{Results}

% Because of the COVID-19 pandemic no user testing has been done this semester, in fact no medical students or professionals have tried the application in-person. Thus, it has been difficult to do formal interviews or gather much feedback, especially regarding interaction. This project is the preliminary work for my master thesis next semester and result gathering will naturally be a much more in focus then. And though no user testing has found place, we have arranged live demoes with \nameref{chap:wdp} over Zoom, which have generated useful feedback. 
% In one such demonstration, I wore a HoloLens 2 and use Nevrolens with guidance from a neuroscientist to extract related regions of the brain and was lectured on their role in behavior. 
% The feedback on its use for a single user, was that there should be a global list menu to toggle different features on each brain part, that there should be ability to increase resolution of a single brain part and some way to visualize microscopical data. 

% % Mostly, the feedback has been positive

% % Success in picking out brain parts, and explaining different structures in the brain. 



% % results from this project are limited.
\chapter{Discussion}



\section{Performance}\label{chap:discussperformance}

The main limiting factor for the performance of the Nevrolens application is the polygon count for the models used. The count for the models uses are around 300,000 for the complete model at decimation ratio 0.08 and an hollow outline model with decimation ration 0.4 at about 350,000 polygons, used in clustering. The counts are higher than what is generally recommended for HoloLens 2, naturally this results in an application which runs well under the best recommended performance quality criteria which the non-functional requirements of this research project set out to meet. The application will hover around 40 to 50 frames per second while in normal use and drop to around 30 to 35 while rendering the volumetric dissection plane, the recommended quality criteria lists 60 fps as a goal for \textit{Best} performance, but the application follows the \textit{Meets} criteria, which state that:
\begin{quote}
\textit{The app has intermittent frame drops not impeding the core experience, or FPS is consistently lower than desired goal but doesn’t impede the app experience.}
\end{quote}
Thus, the application can be said to meet the quality criteria set by Microsoft. 
Throughout testing, framerate was genuinely not a detectable issue, no user tester did at any time commented on low performance and even testers experienced with HoloLens and AR technology did not unprompted notice the low framerate while using the application. This could be the result of the somewhat blurry display of the HoloLens 2, or simply that targeting 60 fps is far less critical in AR application, than in on VR devices. 
On tested Android devices performance has not been of any concern, running at a consistent 60 fps all the time. The teste device is a Samsung Galaxy S8, which was released in 2017 way before the HoloLens 2, and does in fact have a slower processing unit than the HoloLens 2 does. Answering why this performance gap exist will only be speculation, and will not be attempted.
In conclusion, the performance of the application is at an acceptable level and does not impede the user experience, if that were to happen, either by increased load from demanding features or porting to less performant platforms, a natural and easy solution would be to scale down the brain model further.

\section{Hardware limitations}

\subsection*{HoloLens 2 display technology}

\subsection*{Android interaction}
% Sterioscopic rendering

\subsection*{Android spatial locking}

\section{Missing data set}
The brain model data set used in this research, which as described in \autoref{chap:zeroiter} was imported from \nameref{chap:vrvis}, is missing a number of brain structures. The model consists of 29 separate structures while the Waxholm Space Rat Brain model it is based on included as least 75 structures. This is of course a huge discrepancy, but it is not as critical as it could seem as most of the omitted structures are very small and would be impractical to handle within AR. Upon questioning Elden has explained that the reduction was done to decrease computational time as the model was meant for a proof of concept. As this is still the case the reduces number of brain structures is still not a critical issue, however if this research project were to be used as an actual educational tool this would be of most dire need for fixing. Another reason this issue has not seen any need for fixing is that there will be a new version of the Waxholm Space brain model released shortly, and stakeholders at Kavli are interested in the use of this model in further research, thus creation of a new brain model has been deferred to \nameref{chap:futurework}.

% There is however one quite notable structure missing, the \textit{Corpus Collosum} .
\section{Human neuroanatomy}
Describe/explore the feasibility to implement the system for Human neuroanatomy education.
% display issues

\section{Contribution}

\section{Limitations}
% !TEX encoding = UTF-8 Unicode
%!TEX root = thesis.tex
% !TEX spellcheck = en-US
%%=========================================
\chapter{Conclusion}




%%=========================================
\section{Limitations}

This project rapport is based on work from a 15 credits course, thus there has been time constraints managing the project with other courses and exams. Combined with the busy schedule of the neuroscientists at the Kavli Institute, it has been challenging to gather feedback in the relatively short time frame of this project.
Of course, the COVID-19 pandemic has made physical sharing of the headset problematic, and thus user testing has been difficult. 
In total, there is a very thin basis for results in this project, but this will hopefully be turned around for the master project.

%%=========================================
\section{Conclusion}

%%=========================================
\section{Discussion}

% The research question the project set out to answer are


This project has mainly focused on implementing the application answering \href{subrq1}{Sub-RQ1}. As such this can be seen as a minimal viable product for exploring Sub-RQ1, and in that capacity it has shown great promise.



%%=========================================
\section{Further Work}\label{chap:futurework}

\begin{itemize}
    \item Collaboration, Networking
    \item Macro+micro implementation
    \item Importing new models
    \item Look into volumetric rendering  
\end{itemize}

% Include more chapters as required.
%%=========================================
\appendix
% !TEX encoding = UTF-8 Unicode
%!TEX root = thesis.tex
% !TEX spellcheck = en-US
%%=========================================

\chapter{Acronyms}
\begin{description}
\item[NTNU] Norwegian University of Science and Technology
\item[AR] Augmented Reality
\item[MR] Mixed Reality
\item[XR] Extended Reality
\item[VR] Virtual Reality
\item[HMD] Head-mounted display
\item[GPU] Graphics Processing Unit
\item[SDK] Software Development Kit
\item[COVID-19] Coronavirus Disease 2019
\end{description}
% !TEX encoding = UTF-8 Unicode
%!TEX root = thesis.tex
% !TEX spellcheck = en-US
%%=========================================

\chapter{A geometric model of the rat brain}

This is a section from \citet{Elden2017}, the master thesis about \nameref{chap:vrvis} the VR application this project is loosely inspired by. The section explains how Elden extracted a geometric model of the rat brain from the medical models which is high fidelity volumetric data.

\section*{5.2 Exporting segments of a rat brain atlas as geometry
for the Rat Brain model}
The geometric meshes used for the rat brain model were extracted from
a volumetric and segmented atlas. IKT-SNAP was used to export each
segment of the brain as an STL file. These geometric meshes were then
opened in Blender3
to be converted to OBJ or FBX files. 3DS Max imported
the models and performed all modifications made to the geometry and
structure.
ITK-SNAP requires three files to segment and label the models; the
atlas and a segmentation file, both stored as NII files, and a LABEL file
for the labels. When all files are loaded the program lets the user select a
segment to export and generate a geometric hull along the boundary of the
segment. Due to instability experienced with 3DS Max using all 16 GB of
RAM available on the computer used for development, Blender was used
to first convert the files to FBX files. These FBX files caused no issues when
imported into 3DS Max. Since these meshes were too detailed, they needed
to be reduced and transformed in 3DS Max. The meshes were reduced
such that the entire model consisted of 4.5 million triangles. Most of the
meshes had to be transformed such that each segment was where it should
be inside the model. For some reason the exported meshes were of several
relative scales and heights and a lot of manual work went into moving
and scaling the meshes to match the volumetric model seen in ITK-SNAP.
Properly processed, the model was exported as an FBX file and sent to UE4.

\chapter{Neuroanatomical test}\label{chap:knowtest}

The following is the neurological knowledge test used in the research, prior to it a table with the solution and the participants answers.


% Please add the following required packages to your document preamble:
% \usepackage{graphicx}
% \usepackage[table,xcdraw]{xcolor}
% If you use beamer only pass "xcolor=table" option, i.e. \documentclass[xcolor=table]{beamer}
\begin{table}[H]
\centering
\resizebox{0.6\textwidth}{!}{%
\begin{tabular}{lll}
\rowcolor[HTML]{9698ED} 
\multicolumn{1}{c}{\cellcolor[HTML]{9698ED}Id} &
  \multicolumn{1}{c}{\cellcolor[HTML]{9698ED}Answers} &
  \multicolumn{1}{c}{\cellcolor[HTML]{9698ED}Score} \\
\rowcolor[HTML]{CBCEFB} 
\multicolumn{3}{c}{\cellcolor[HTML]{CBCEFB}First session} \\
1a &
  \begin{tabular}[c]{@{}l@{}}ccbcc cbadba cbbac badab babad\\ cbbac bbabca cabac baacc aabbb\end{tabular} &
  \begin{tabular}[c]{@{}l@{}}19\\ 17\end{tabular} \\
\rowcolor[HTML]{EFEFEF} 
1b &
  \begin{tabular}[c]{@{}l@{}}cbcba cbdddb cccac babad bdbad\\ cbcaa cbbddd cbcab babab bcbad\end{tabular} &
  \begin{tabular}[c]{@{}l@{}}15\\ 15\end{tabular} \\
1c &
  \begin{tabular}[c]{@{}l@{}}cbaaa cbaddd ccbac bdcad babab\\ ccaaa cbbbdd cbdac bdcad babab\end{tabular} &
  \begin{tabular}[c]{@{}l@{}}18\\ 16\end{tabular} \\
\rowcolor[HTML]{CBCEFB} 
\multicolumn{3}{c}{\cellcolor[HTML]{CBCEFB}Second session} \\
2a &
  \begin{tabular}[c]{@{}l@{}}ccdac bbbddd ccaac badad babca\\ ccaac bbbdda ccaac babad babcb\end{tabular} &
  \begin{tabular}[c]{@{}l@{}}13\\ 14\end{tabular} \\
\rowcolor[HTML]{EFEFEF} 
2b &
  \begin{tabular}[c]{@{}l@{}}cbdad dbbddd ccaac baadd bbddd\\ cbaac cbaddd ccbac baaab bbdad\end{tabular} &
  \begin{tabular}[c]{@{}l@{}}10\\ 17\end{tabular} \\
2c & \begin{tabular}[c]{@{}l@{}}ccdaa cbaddd ccbac baadd abbdd\\ ccbaa ccbaddd ccbac baabd bbbba\end{tabular} & \begin{tabular}[c]{@{}l@{}}12\\ 13\end{tabular} \\
\rowcolor[HTML]{EFEFEF} 
2d &
  \begin{tabular}[c]{@{}l@{}}cbdaa bdbddd ccaac bbdab cbcbb\\ ccaaa bbaddd ccacc bbadb babab\end{tabular} &
  \begin{tabular}[c]{@{}l@{}}10\\ 13\end{tabular} \\
2e &
  \begin{tabular}[c]{@{}l@{}}dddad acbddd cdaad daddd bccad\\ ccaac acaddd cccac baaad bbbcc\end{tabular} &
  \begin{tabular}[c]{@{}l@{}}6\\ 13\end{tabular} \\
\rowcolor[HTML]{CBCEFB} 
\multicolumn{3}{c}{\cellcolor[HTML]{CBCEFB}Solution} \\
 &
  cbcac cbabbb cabac bacab babab &
  26
\end{tabular}%
}
\end{table}

\includepdf[pages={1, 3-6}, angle=90]{appendix/knowtest}
\chapter{Nevrolens Questionnaire}\label{chap:questionnaire}
\chapter{Consent Form}\label{appendix:consent}


\includepdf[pages=-]{appendix/consentform}
% Include more appendices as required.
%%=========================================
\bibliographystyle{apa}
\addcontentsline{toc}{chapter}{\bibname}
\bibliography{refs}  
%%=========================================
%\include{vitae}         % Your curriculum Vitae     
%%=============================================

\end{document}
