% !TEX encoding = UTF-8 Unicode
% !TEX root = thesis.tex
% !TEX spellcheck = en-US
%%=========================================
\addcontentsline{toc}{section}{Abstract}
\section*{Abstract}

% The following rapport is part of TDT4501 Specialization Project

This study aims to explore whether Augmented Reality can be used as a tool for medical students learning neuroanatomy. A application, \textit{Nevrolens}, was created with features approximating a conventional rat brain dissection, as well as a collaboration tools for students and educators to cooperate to simulate a lecturing environment. 
The thesis will explore the implementation of such a system, as well as how it performs in educational settings, and its possible use as a tool for remote learning.
While the results from this study are limited, they indicate the application to be of value in educating lower level medical students and that a AR system of this scope can be simple in use even for users with no prior experience with AR devices.
\newline

\textit{
\textbf{Keywords:} Augmented Reality, Mixed Reality, Education, Collaboration, Neuroanatomy, Remote learning
}

% the application has shown promise in its use as a virtual simulation of a rat brain dissection.