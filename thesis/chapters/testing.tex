\chapter{Testing}


This chapter will explore how testing has been accomplished in this research project, both day-to-day software testing, but also feedback from stakeholders and formal user testing. Because of the current COVID-19 pandemic, physical meeting and user testing been done sparingly, and have at times been impossible carry out, in fact the fall semester did not see any physical testing by users not affiliated with the VRLab at NTNU Dragvoll.

\section{Software testing}
During development, software testing has been done unstructured, mostly by way of regular deployment and on-device feature testing. This is admittedly not the most comprehensive testing system and can not guarantee intended behavior as systems are combined and restructured, in the same way as a test suit with \textit{unit testing} and \textit{integration testing} could. But it allows for more rapid development, less overhead for a single developer and can not be said to have meaningfully limited the research project. A case could be made for a testing suit being helpful in the continuation of the research as new developers will have a concrete indication of intended behavior, this has and could not have a high priority as a development goal, because of limitations in development time and the demand for creating a research result.

\section{User Testing Precautions}

\subsection*{Consent}
Every user tester where handed a consent form at the beginning of each session. This assured the privacy rights of the test persons, and gave approval to use the findings of the test session in this research. The consent form can be found in \autoref{appendix:conent}.

\subsection*{Hygienic Measures}
Due to the COVID-19 pandemic, many new precautions have been enacted to prevent the spread of virtual pathogens. The most effective measure has undoubtedly been the limited number of testing sessions, even so what follow are the precautions taken during those physical meetings. 
Firstly, the general national guidelines of social distancing and clean hands were kept. 
Between each new person using a head mounted HoloLens 2, the headset was placed in a \textit{Cleanbox}, this is system specially design to disinfect AR and VR devices utilizing UVC radiation to destroy viruses. It claims to kill Covid-19 viruses with 99.999+\% efficacy\footnote{https://cleanboxtech.com}. Further, contact points and buttons on the device are swiped with alcohol-based sanitizer. The reason this step was not enough and the UVC box was needed is because of the fragile nature of the lenses of the AR device, it is generally recommended not to touch the lenses, and alcohol would naturally not be good. On Android devices this was however the main disinfection method as cellphones could be completely wiped. Lastly, the user would ware a hygienic mask under the headset which reduced contact between the user and the headset.
All in all these actions in combination with strict adherence to the national guidelines of physical distancing, washing and sanitizing of hands and waring of face masks were done to prevent spread of the virus. 

% zorro masks, UVC radiation, anti bacterial desinfactant, munnbind og avstand

\section{Stakeholder meetings}
Throughout the project, there have been multiple meetings with stakeholders from the Kavli Institute at St.Olavs. These have been a combination of physical meetings and virtual video conferences using Zoom. When meeting virtually the research prepared one or more video demonstrations captured on-device, either HoloLens 2 or Android, which were uploaded to YouTube for convenience. When meeting physically either the researcher or the stakeholder would use the application with live view enabled such that the other could see and comment on the usage. Afterwards, the progress was discussed and evaluated and feedback was given on usability and features. Primarily, the resulting feedback was in the form of feature request, and general technical in nature.
% Because of differing backgrounds the researcher and stakeholders had some communication issues,

\section{User Testing}
During the final stages of the research project, two user testing sessions were held. 

The first user testing had three participants, it began by them agreeing to a consent forum and then taking a test of their knowlege


and was somewhat unstructured in its execution. This resulted in 