% !TEX encoding = UTF-8 Unicode
% !TEX root = thesis.tex
% !TEX spellcheck = en-US
%%=========================================



\chapter{Implementation}

% Hololens, UWP, WMR

\section{Requirements}

% cadavers difficult to get, vr -> less awareness
%
%


The first meeting initializing the project happened at VRLab Dragvoll in early September, here I was introduced to the general background and the problem description of how Witter and others envisioned the use of AR for neuroanatomical education. It was explained how cadavers for education are difficult to acquire and [\dots]. 
Another problem we discussed was related to the difference in medium between VR and AR. While the application \nameref{chap:vrvis} did have many of the features envisioned, the fact that is was implemented in VR was problematic for the envisioned use cases. Being completely enclosed visually limits its use case in lectures and in any use case with collaboration in a physical space. Generally the loss of spatial awareness and eye contact as a result of using VR headsets was though of as an impediment for using VR for such an application. 
Thus, we had an outline of a neuroanatomical education tool in AR using the HoloLens and 





\section{Software Process}
% https://www.wikipendium.no/TDT4140_Programvareutvikling/nb/#kapittel-2-software-processes
Incremental development etc. \#agile

Even though I have developed the Nevrolens application by myself, I have strived to use best practices for a software development workflow. These practices have generally grown out of the the needs of a multi-developer setting, to simplify the needs for collaboration and version control. Though their value possibly increases exponentially by the number of team members, I have found value in the structure and clarity I find in the workflow. 
My workflow consist of 



\large{Where everything I learned in PU should shine!}

\section{Validation / Testing}


