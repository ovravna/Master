\chapter{Technical Design}

This chapter will give a overview of the structure of the application as well as some choices taken when developing the research product, Nevrolens. 

\subsection*{Unity Game structure}

Within Unity a \textit{Scene} consists of a \textit{scene graph} which is a tree structure of \texttt{GameObjects}. By default a scene consist of a \texttt{Directional Light} lighting up the scene at its default light source and a \texttt{Main Camera} which is the view point of the running game. In addition, the MRTK library will add two objects to the scene graph, one called \texttt{MixedRealityToolkit} which contains configuration of the Mixed Reality features and systems. This is where input systems are defined and where control of spatial awareness and boundary detection is handled, in short all features and sensors of the HoloLens system or other AR system are defined and controlled here. The other object added by MRTK is the \texttt{MixedRealityPlayspace}, this encapsulates the \texttt{Main Camera}, but is lacking any useful documentation on what its purpose is. The name could be hinting at it being the parent of the \textit{Playspace}, meaning all \texttt{GameObjects} in the game. However, even MRTK demos seem to ignore this object and thus it has not been used in this project either.

The functionality of the scene graph, other than organizing GameObjects, is that child objects inherit the position, rotation and scale of their parents, thus simplifying transformation of complex object. This naturally structures many systems, however in a AR application there can be many independent 3D objects floating in space. In addition, some objects are dependent not on their parent, but on a defined object \texttt{Transform}. Therefore, some organization is needed and some objects are placed by choice and convenience rather than any practical reason.

The main object of the application is the \texttt{BrainSystem}, this acts as the parent GameObject for all objects defined by the application. 
Thus, natural

% In general there is few limitations on how to structure the scene graph, its functionality other than structural,
