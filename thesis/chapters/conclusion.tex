% !TEX encoding = UTF-8 Unicode
%!TEX root = thesis.tex
% !TEX spellcheck = en-US
%%=========================================
\chapter{Conclusion}




%%=========================================
\section{Limitations}

This project rapport is based on work from a 15 credits course, thus there has been time constraints managing the project with other courses and exams. Combined with the busy schedule of the neuroscientists at the Kavli Institute, it has been challenging to gather feedback in the relatively short time frame of this project.
Of course, the COVID-19 pandemic has made physical sharing of the headset problematic, and thus user testing has been difficult. 
In total, there is a very thin basis for results in this project, but this will hopefully be turned around for the master project.

%%=========================================
\section{Conclusion}

%%=========================================
\section{Discussion}

% The research question the project set out to answer are


This project has mainly focused on implementing the application answering \href{subrq1}{Sub-RQ1}. As such this can be seen as a minimal viable product for exploring Sub-RQ1, and in that capacity it has shown great promise.



%%=========================================
\section{Further Work}\label{chap:futurework}

\begin{itemize}
    \item Collaboration, Networking
    \item Macro+micro implementation
    \item Importing new models
    \item Look into volumetric rendering  
\end{itemize}
