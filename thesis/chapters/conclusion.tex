% !TEX encoding = UTF-8 Unicode
%!TEX root = thesis.tex
% !TEX spellcheck = en-US
%%=========================================
\chapter{Conclusion}




%%=========================================
\section{Limitations}

This project rapport is based on work from a 15 credits course, thus there has been time constraints managing the project with other courses and exams. Combined with the busy schedule of the neuroscientists at the Kavli Institute, it has been challenging to gather feedback in the relatively short time frame of this project.
Of course, the COVID-19 pandemic has made physical sharing of the headset problematic, and thus user testing has been difficult. 
In total, there is a very thin basis for results in this project, but this will hopefully be turned around for the master project.


%%=========================================
\section{Discussion}

% The research question the project set out to answer are


This project has mainly focused on implementing the application answering \href{subrq1}{Sub-RQ1}. As such this can be seen as a minimal viable product for exploring Sub-RQ1, and in that capacity it has shown great promise. The basic tools for dissection is implemented and the medical academics at Kavli Institute and UiO sees great potential in its use. 
As for the Sub-RQ2 and 3, there is less work done. However, by building the application for Android, I have shown that the possibility for cross-platform collaboration is open. And it will be worked on for the master thesis, by implementing networking with \nameref{chap:photon}. Sub-RQ3, has not seen any concrete development, I do have some ideas of how it could be implemented with textures generated from a volumetric brain model. 


%%=========================================
\section{Conclusion}

The state of the project now is a small scale proof of concept for a AR application supporting teaching of neuroanatomy and dissection for medical students. There is still no collaboration or microscopic data visualization in the app. However, the biggest problem with the state of the project is the lack of concrete feedback from medical user groups. It has been received very positively by the neuroscientists who has seen the application in use, and they are awaiting more news on the project. The need for exact user testing and user feedback is still large. The project is however progressing nicely, and with further development I believe this project could create impressing results.


%%=========================================
\section{Further Work}\label{chap:futurework}

% \begin{itemize}
%     \item Collaboration, Networking
%     \item Macro+micro implementation
%     \item Importing new models
%     \item Look into volumetric rendering  
% \end{itemize}

There is still a lot of work left on this project, and it will be continues by me. While there is numerous features and fixes which are waiting in the backlog I would like to focus on the overall picture and what has to be in focus to create an collaborative educational experience in AR. 

\begin{itemize}

    \item {
        \textbf{Focus on usability}\\
        Implement better guidance and affordance such that anyone can use the application. To manage this it will be essential to have user testers and testing with medical user groups. This will give feedback on what makes sense and what doesn't.
        
    }
    \item {
        \textbf{Collaboration; implement cross-platform networking}\\
        Networking tools will be required to create a collaborative experience, this will be done using \nameref{chap:photon} to synchronize the HoloLens 2 application with the Android application.
    }

    \item {
        \textbf{Visualize volumetric data}\\
        While the HoloLens 2 does support volumetric rendering, I am pretty sure it is not in an adequate resolution for this project. It will however be explored. Other than that volumetric data could be used as 2D textures mapped on the clipping plane used to cut the brain.
    }

    \item {
        \textbf{Explore ways of using other brains}\\
        \autoref{chap:elden} explains the steps taken by \citet{Elden2017} to use medical imagery in \nameref{chap:vrvis}. An exploration into whether there is a more elegant way of doing this, should be done.
    }

A general focus when continuing this project should be gathering more user data in the form of user tests and interviews, there should be a focus on establishing whether this project can improve learning outcomes for medical students.


\end{itemize}